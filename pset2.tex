% Options for packages loaded elsewhere
\PassOptionsToPackage{unicode}{hyperref}
\PassOptionsToPackage{hyphens}{url}
%
\documentclass[
]{article}
\usepackage{amsmath,amssymb}
\usepackage{iftex}
\ifPDFTeX
  \usepackage[T1]{fontenc}
  \usepackage[utf8]{inputenc}
  \usepackage{textcomp} % provide euro and other symbols
\else % if luatex or xetex
  \usepackage{unicode-math} % this also loads fontspec
  \defaultfontfeatures{Scale=MatchLowercase}
  \defaultfontfeatures[\rmfamily]{Ligatures=TeX,Scale=1}
\fi
\usepackage{lmodern}
\ifPDFTeX\else
  % xetex/luatex font selection
\fi
% Use upquote if available, for straight quotes in verbatim environments
\IfFileExists{upquote.sty}{\usepackage{upquote}}{}
\IfFileExists{microtype.sty}{% use microtype if available
  \usepackage[]{microtype}
  \UseMicrotypeSet[protrusion]{basicmath} % disable protrusion for tt fonts
}{}
\makeatletter
\@ifundefined{KOMAClassName}{% if non-KOMA class
  \IfFileExists{parskip.sty}{%
    \usepackage{parskip}
  }{% else
    \setlength{\parindent}{0pt}
    \setlength{\parskip}{6pt plus 2pt minus 1pt}}
}{% if KOMA class
  \KOMAoptions{parskip=half}}
\makeatother
\usepackage{xcolor}
\usepackage[margin=1in]{geometry}
\usepackage{color}
\usepackage{fancyvrb}
\newcommand{\VerbBar}{|}
\newcommand{\VERB}{\Verb[commandchars=\\\{\}]}
\DefineVerbatimEnvironment{Highlighting}{Verbatim}{commandchars=\\\{\}}
% Add ',fontsize=\small' for more characters per line
\usepackage{framed}
\definecolor{shadecolor}{RGB}{248,248,248}
\newenvironment{Shaded}{\begin{snugshade}}{\end{snugshade}}
\newcommand{\AlertTok}[1]{\textcolor[rgb]{0.94,0.16,0.16}{#1}}
\newcommand{\AnnotationTok}[1]{\textcolor[rgb]{0.56,0.35,0.01}{\textbf{\textit{#1}}}}
\newcommand{\AttributeTok}[1]{\textcolor[rgb]{0.13,0.29,0.53}{#1}}
\newcommand{\BaseNTok}[1]{\textcolor[rgb]{0.00,0.00,0.81}{#1}}
\newcommand{\BuiltInTok}[1]{#1}
\newcommand{\CharTok}[1]{\textcolor[rgb]{0.31,0.60,0.02}{#1}}
\newcommand{\CommentTok}[1]{\textcolor[rgb]{0.56,0.35,0.01}{\textit{#1}}}
\newcommand{\CommentVarTok}[1]{\textcolor[rgb]{0.56,0.35,0.01}{\textbf{\textit{#1}}}}
\newcommand{\ConstantTok}[1]{\textcolor[rgb]{0.56,0.35,0.01}{#1}}
\newcommand{\ControlFlowTok}[1]{\textcolor[rgb]{0.13,0.29,0.53}{\textbf{#1}}}
\newcommand{\DataTypeTok}[1]{\textcolor[rgb]{0.13,0.29,0.53}{#1}}
\newcommand{\DecValTok}[1]{\textcolor[rgb]{0.00,0.00,0.81}{#1}}
\newcommand{\DocumentationTok}[1]{\textcolor[rgb]{0.56,0.35,0.01}{\textbf{\textit{#1}}}}
\newcommand{\ErrorTok}[1]{\textcolor[rgb]{0.64,0.00,0.00}{\textbf{#1}}}
\newcommand{\ExtensionTok}[1]{#1}
\newcommand{\FloatTok}[1]{\textcolor[rgb]{0.00,0.00,0.81}{#1}}
\newcommand{\FunctionTok}[1]{\textcolor[rgb]{0.13,0.29,0.53}{\textbf{#1}}}
\newcommand{\ImportTok}[1]{#1}
\newcommand{\InformationTok}[1]{\textcolor[rgb]{0.56,0.35,0.01}{\textbf{\textit{#1}}}}
\newcommand{\KeywordTok}[1]{\textcolor[rgb]{0.13,0.29,0.53}{\textbf{#1}}}
\newcommand{\NormalTok}[1]{#1}
\newcommand{\OperatorTok}[1]{\textcolor[rgb]{0.81,0.36,0.00}{\textbf{#1}}}
\newcommand{\OtherTok}[1]{\textcolor[rgb]{0.56,0.35,0.01}{#1}}
\newcommand{\PreprocessorTok}[1]{\textcolor[rgb]{0.56,0.35,0.01}{\textit{#1}}}
\newcommand{\RegionMarkerTok}[1]{#1}
\newcommand{\SpecialCharTok}[1]{\textcolor[rgb]{0.81,0.36,0.00}{\textbf{#1}}}
\newcommand{\SpecialStringTok}[1]{\textcolor[rgb]{0.31,0.60,0.02}{#1}}
\newcommand{\StringTok}[1]{\textcolor[rgb]{0.31,0.60,0.02}{#1}}
\newcommand{\VariableTok}[1]{\textcolor[rgb]{0.00,0.00,0.00}{#1}}
\newcommand{\VerbatimStringTok}[1]{\textcolor[rgb]{0.31,0.60,0.02}{#1}}
\newcommand{\WarningTok}[1]{\textcolor[rgb]{0.56,0.35,0.01}{\textbf{\textit{#1}}}}
\usepackage{graphicx}
\makeatletter
\newsavebox\pandoc@box
\newcommand*\pandocbounded[1]{% scales image to fit in text height/width
  \sbox\pandoc@box{#1}%
  \Gscale@div\@tempa{\textheight}{\dimexpr\ht\pandoc@box+\dp\pandoc@box\relax}%
  \Gscale@div\@tempb{\linewidth}{\wd\pandoc@box}%
  \ifdim\@tempb\p@<\@tempa\p@\let\@tempa\@tempb\fi% select the smaller of both
  \ifdim\@tempa\p@<\p@\scalebox{\@tempa}{\usebox\pandoc@box}%
  \else\usebox{\pandoc@box}%
  \fi%
}
% Set default figure placement to htbp
\def\fps@figure{htbp}
\makeatother
\setlength{\emergencystretch}{3em} % prevent overfull lines
\providecommand{\tightlist}{%
  \setlength{\itemsep}{0pt}\setlength{\parskip}{0pt}}
\setcounter{secnumdepth}{-\maxdimen} % remove section numbering
\usepackage{comment}
\newcommand{\BetaDist}{\mathrm{Beta}}
\newcommand{\Binom}{\mathrm{Binomial}}
\newcommand{\E}{\mathbb{E}}
\newcommand{\Prob}{\mathbb{P}}
\usepackage{bookmark}
\IfFileExists{xurl.sty}{\usepackage{xurl}}{} % add URL line breaks if available
\urlstyle{same}
\hypersetup{
  pdftitle={Math for Data Science: Problem Set 2},
  pdfauthor={Group: GROUP\_MEMBERS\_NAMES},
  hidelinks,
  pdfcreator={LaTeX via pandoc}}

\title{Math for Data Science: Problem Set 2}
\author{Group: GROUP\_MEMBERS\_NAMES}
\date{2025-11-11}

\begin{document}
\maketitle

\textbf{Due Date:} Thursday, November 20 by the end of the day. (The
Moodle submission link will become inactive at midnight of November 21.)

\textbf{Instructions:} Please submit one solution set per group and
include your group members' names at the top. This time, please write
your solutions within this Rmd file, under the relevant question. Please
submit the knitted output as a pdf. Make sure to show all code you used
to arrive at the answer. However, please provide a brief, clear answer
to every question rather than making us infer it from your output, and
please avoid printing unnecessary output.

\subsection{1. The Law of Large Numbers and the Central Limit
Theorem}\label{the-law-of-large-numbers-and-the-central-limit-theorem}

You are an urban planner interested in finding out how many people enter
and leave the city using personal vehicles every day. (You're not
interested in the number of \emph{cars}; you're interested in the number
of \emph{people} who use cars to get to work.) To do this, you decide to
collect data from a few different points around the city on how many
people there are per car. You already have reliable satellite data on
the number of cars that come into the city, so if you get a good
estimate of people per car you'll be in good shape.

Collecting data on people per car is costly and you'd love to minimize
how many data points you have to collect. However, you're also familiar
with the Law of Large Numbers and know that the sample mean converges to
the true mean as the sample size \(n\) grows large.

\begin{enumerate}
\def\labelenumi{\alph{enumi}.}
\tightlist
\item
  Let's illustrate this with a small simulation. Suppose the number of
  people in a car is distributed Poisson with a rate of \(\lambda=2\)
  people per car.\footnote{I should have mentioned that you're an urban
    planner in San Francisco, where it's rare but possible to have 0
    people in a car.} Construct 500 samples from this distribution, with
  the first sample having \(n=1\) cars, the second \(n=2\) cars, and so
  on. Compute the average number of people per car in each sample. Plot
  this on the y-axis against the sample size on the x-axis and run a
  horizontal blue line through the true mean. Comment on what you see.
\end{enumerate}

\begin{Shaded}
\begin{Highlighting}[]
\CommentTok{\# Set parameters}
\FunctionTok{set.seed}\NormalTok{(}\DecValTok{666}\NormalTok{)}
\NormalTok{lambda }\OtherTok{\textless{}{-}} \DecValTok{2}
\NormalTok{max\_n }\OtherTok{\textless{}{-}} \DecValTok{500}

\CommentTok{\# Generate sample means for each n}
\NormalTok{sample\_means }\OtherTok{\textless{}{-}} \FunctionTok{sapply}\NormalTok{(}\DecValTok{1}\SpecialCharTok{:}\NormalTok{max\_n, }\ControlFlowTok{function}\NormalTok{(n) \{}
  \FunctionTok{mean}\NormalTok{(}\FunctionTok{rpois}\NormalTok{(n, lambda))}
\NormalTok{\})}

\CommentTok{\# Plot results}
\FunctionTok{plot}\NormalTok{(}\DecValTok{1}\SpecialCharTok{:}\NormalTok{max\_n, sample\_means, }\AttributeTok{type =} \StringTok{"l"}\NormalTok{,}
     \AttributeTok{xlab =} \StringTok{"Sample size (n)"}\NormalTok{,}
     \AttributeTok{ylab =} \StringTok{"Average people per car"}\NormalTok{,}
     \AttributeTok{main =} \StringTok{"Law of Large Numbers: Sample Mean of Poisson(λ = 2)"}\NormalTok{)}
\FunctionTok{abline}\NormalTok{(}\AttributeTok{h =}\NormalTok{ lambda, }\AttributeTok{col =} \StringTok{"blue"}\NormalTok{, }\AttributeTok{lwd =} \DecValTok{2}\NormalTok{, }\AttributeTok{lty =} \DecValTok{2}\NormalTok{)}
\end{Highlighting}
\end{Shaded}

\pandocbounded{\includegraphics[keepaspectratio]{pset2_files/figure-latex/unnamed-chunk-1-1.pdf}}

\begin{enumerate}
\def\labelenumi{\alph{enumi}.}
\setcounter{enumi}{1}
\item
  You collect data on 100 cars and compute the average number of people
  per car in this sample. Use the Central Limit Theorem to write down
  the approximate distribution of this quantity.
\item
  Let's examine this distribution more closely. Generate 10,000
  replicates of the sample mean with \(n=100\) and plot a
  histogram.\footnote{Try using the \texttt{replicate} function rather
    than a loop, as this will speed things up considerably.} Are you
  convinced that the Normal approximation you found in the previous
  question is good enough? Compare this to \(n=1\), \(n=5\), and
  \(n=30\), generating a histogram for each. (We're aiming to recreate
  the second row of Figure 10.5 from Slide 47 of Lecture 4.) Comment on
  what you observe.
\end{enumerate}

\begin{Shaded}
\begin{Highlighting}[]
\FunctionTok{library}\NormalTok{(ggplot2)}

\CommentTok{\# Setup}
\FunctionTok{set.seed}\NormalTok{(}\DecValTok{666}\NormalTok{)}
\NormalTok{lambda }\OtherTok{\textless{}{-}} \DecValTok{2}
\NormalTok{reps   }\OtherTok{\textless{}{-}} \DecValTok{10000}
\NormalTok{Ns     }\OtherTok{\textless{}{-}} \FunctionTok{c}\NormalTok{(}\DecValTok{1}\NormalTok{, }\DecValTok{5}\NormalTok{, }\DecValTok{30}\NormalTok{, }\DecValTok{100}\NormalTok{)}

\CommentTok{\# Simulate sample means using replicate()}
\NormalTok{sim\_mat }\OtherTok{\textless{}{-}} \FunctionTok{sapply}\NormalTok{(Ns, }\ControlFlowTok{function}\NormalTok{(n) }\FunctionTok{replicate}\NormalTok{(reps, }\FunctionTok{mean}\NormalTok{(}\FunctionTok{rpois}\NormalTok{(n, lambda))))}
\FunctionTok{colnames}\NormalTok{(sim\_mat) }\OtherTok{\textless{}{-}} \FunctionTok{paste0}\NormalTok{(}\StringTok{"n="}\NormalTok{, Ns)}

\CommentTok{\# Tidy for plotting}
\NormalTok{df }\OtherTok{\textless{}{-}} \FunctionTok{data.frame}\NormalTok{(}
  \AttributeTok{mean =} \FunctionTok{as.vector}\NormalTok{(sim\_mat),}
  \AttributeTok{n    =} \FunctionTok{factor}\NormalTok{(}\FunctionTok{rep}\NormalTok{(Ns, }\AttributeTok{each =}\NormalTok{ reps), }\AttributeTok{levels =}\NormalTok{ Ns)}
\NormalTok{)}

\CommentTok{\# Normal curves to overlay}
\NormalTok{xgrid }\OtherTok{\textless{}{-}} \FunctionTok{seq}\NormalTok{(}\DecValTok{0}\NormalTok{, }\DecValTok{5}\NormalTok{, }\AttributeTok{by =} \FloatTok{0.01}\NormalTok{)}
\NormalTok{norm\_curves }\OtherTok{\textless{}{-}} \FunctionTok{do.call}\NormalTok{(rbind, }\FunctionTok{lapply}\NormalTok{(Ns, }\ControlFlowTok{function}\NormalTok{(n) \{}
  \FunctionTok{data.frame}\NormalTok{(}
    \AttributeTok{x =}\NormalTok{ xgrid,}
    \AttributeTok{y =} \FunctionTok{dnorm}\NormalTok{(xgrid, }\AttributeTok{mean =}\NormalTok{ lambda, }\AttributeTok{sd =} \FunctionTok{sqrt}\NormalTok{(lambda }\SpecialCharTok{/}\NormalTok{ n)),}
    \AttributeTok{n =} \FunctionTok{factor}\NormalTok{(n, }\AttributeTok{levels =}\NormalTok{ Ns)}
\NormalTok{  )}
\NormalTok{\}))}

\FunctionTok{ggplot}\NormalTok{(df, }\FunctionTok{aes}\NormalTok{(}\AttributeTok{x =}\NormalTok{ mean)) }\SpecialCharTok{+}
  \FunctionTok{geom\_histogram}\NormalTok{(}\FunctionTok{aes}\NormalTok{(}\AttributeTok{y =} \FunctionTok{after\_stat}\NormalTok{(density)), }\AttributeTok{bins =} \DecValTok{40}\NormalTok{, }\AttributeTok{fill =} \StringTok{"lightgray"}\NormalTok{, }\AttributeTok{color =} \StringTok{"black"}\NormalTok{) }\SpecialCharTok{+}
  \FunctionTok{geom\_line}\NormalTok{(}\AttributeTok{data =}\NormalTok{ norm\_curves, }\FunctionTok{aes}\NormalTok{(}\AttributeTok{x =}\NormalTok{ x, }\AttributeTok{y =}\NormalTok{ y), }\AttributeTok{linewidth =} \DecValTok{1}\NormalTok{, }\AttributeTok{color =} \StringTok{"blue"}\NormalTok{) }\SpecialCharTok{+}
  \FunctionTok{facet\_wrap}\NormalTok{(}\SpecialCharTok{\textasciitilde{}}\NormalTok{ n, }\AttributeTok{nrow =} \DecValTok{2}\NormalTok{) }\SpecialCharTok{+}
  \FunctionTok{coord\_cartesian}\NormalTok{(}\AttributeTok{xlim =} \FunctionTok{c}\NormalTok{(}\DecValTok{0}\NormalTok{, }\DecValTok{5}\NormalTok{)) }\SpecialCharTok{+}
  \FunctionTok{labs}\NormalTok{(}\AttributeTok{title =} \StringTok{"Sampling distribution of the sample mean (Poisson λ = 2)"}\NormalTok{,}
       \AttributeTok{x =} \StringTok{"Sample mean"}\NormalTok{, }\AttributeTok{y =} \StringTok{"Density"}\NormalTok{) }\SpecialCharTok{+}
  \FunctionTok{theme\_minimal}\NormalTok{()}
\end{Highlighting}
\end{Shaded}

\pandocbounded{\includegraphics[keepaspectratio]{pset2_files/figure-latex/unnamed-chunk-3-1.pdf}}

\begin{enumerate}
\def\labelenumi{\alph{enumi}.}
\setcounter{enumi}{3}
\tightlist
\item
  Suppose the city government will enact measures to regulate the number
  of people allowed per car during rush hour if they think the mean is
  below 1.7 people per car. Using the Normal approximation from part (b)
  above, find the probability that you get a mean of 1.7 or less in your
  sample of 100, even though the true mean is 2. (Please give the
  theoretical answer, not a simulation. You can use \texttt{R} as a
  calculator.) What should you do to ensure that this probability stays
  below 1\%?
\end{enumerate}

\begin{Shaded}
\begin{Highlighting}[]
\CommentTok{\# X\_i \textasciitilde{} Poisson(lambda = 2)}
\CommentTok{\# Under the CLT, the sample mean is approximately Normal:}
\CommentTok{\# X̄\_n \textasciitilde{} Normal(mean=lambda, variance = lambda / n)}
\CommentTok{\# For n = 100:}
\NormalTok{lambda }\OtherTok{\textless{}{-}} \DecValTok{2}
\NormalTok{n }\OtherTok{\textless{}{-}} \DecValTok{100}
\NormalTok{mean\_true }\OtherTok{\textless{}{-}}\NormalTok{ lambda}
\NormalTok{sd\_true }\OtherTok{\textless{}{-}} \FunctionTok{sqrt}\NormalTok{(lambda }\SpecialCharTok{/}\NormalTok{ n)}

\CommentTok{\# The city will act if the sample mean equal or less to 1.7}
\NormalTok{threshold }\OtherTok{\textless{}{-}} \FloatTok{1.7}

\CommentTok{\# Compute the probability of observing a mean equal or less 1.7}
\NormalTok{p\_value }\OtherTok{\textless{}{-}} \FunctionTok{pnorm}\NormalTok{(threshold, }\AttributeTok{mean =}\NormalTok{ mean\_true, }\AttributeTok{sd =}\NormalTok{ sd\_true)}
\NormalTok{p\_value}
\end{Highlighting}
\end{Shaded}

\begin{verbatim}
## [1] 0.01694743
\end{verbatim}

\begin{Shaded}
\begin{Highlighting}[]
\CommentTok{\# app. 0.0169 or 1.7\%}

\CommentTok{\# Interpretation output:}
\FunctionTok{cat}\NormalTok{(}
  \StringTok{"Using the CLT, we approximate the sampling distribution of the sample mean as "}\NormalTok{,}
  \StringTok{"Normal(μ = 2, σ² = 2/100). The probability that the observed mean is 1.7 or lower, "}\NormalTok{,}
  \StringTok{"even though the true mean is 2, equals Φ((1.7 {-} 2)/√(2/100)) ="}\NormalTok{,}
  \FunctionTok{round}\NormalTok{(p\_value, }\DecValTok{4}\NormalTok{), }
  \StringTok{", which is about 1.7\%. Thus, there is a 1.7\% chance of falsely concluding the mean is below 1.7.}\SpecialCharTok{\textbackslash{}n}\StringTok{"}
\NormalTok{)}
\end{Highlighting}
\end{Shaded}

\begin{verbatim}
## Using the CLT, we approximate the sampling distribution of the sample mean as  Normal(μ = 2, σ² = 2/100). The probability that the observed mean is 1.7 or lower,  even though the true mean is 2, equals Φ((1.7 - 2)/√(2/100)) = 0.0169 , which is about 1.7%. Thus, there is a 1.7% chance of falsely concluding the mean is below 1.7.
\end{verbatim}

\begin{Shaded}
\begin{Highlighting}[]
\CommentTok{\# To ensure this false{-}alarm probability is below 1\%, we solve for n:}
\NormalTok{target\_alpha }\OtherTok{\textless{}{-}} \FloatTok{0.01}
\NormalTok{z\_alpha }\OtherTok{\textless{}{-}} \FunctionTok{qnorm}\NormalTok{(}\DecValTok{1} \SpecialCharTok{{-}}\NormalTok{ target\_alpha)}
\NormalTok{n\_required }\OtherTok{\textless{}{-}} \FunctionTok{ceiling}\NormalTok{(lambda }\SpecialCharTok{/}\NormalTok{ ((}\FloatTok{0.3} \SpecialCharTok{/}\NormalTok{ z\_alpha)}\SpecialCharTok{\^{}}\DecValTok{2}\NormalTok{))}
\NormalTok{n\_required}
\end{Highlighting}
\end{Shaded}

\begin{verbatim}
## [1] 121
\end{verbatim}

\begin{Shaded}
\begin{Highlighting}[]
\CommentTok{\# Interpretation output:}
\FunctionTok{cat}\NormalTok{(}
  \StringTok{"To reduce this probability below 1\%, we need at least n ="}\NormalTok{,}
\NormalTok{  n\_required,}
  \StringTok{"observations.}\SpecialCharTok{\textbackslash{}n}\StringTok{"}
\NormalTok{)}
\end{Highlighting}
\end{Shaded}

\begin{verbatim}
## To reduce this probability below 1%, we need at least n = 121 observations.
\end{verbatim}

\subsection{2. Maximum Likelihood}\label{maximum-likelihood}

Bangladesh, home to 163 million people, is the world's most populous
delta region; one-fourth of the country's land mass is only seven feet
above sea
level.\footnote{\url{https://www.nrdc.org/stories/bangladesh-country-underwater-culture-move}}
Although the communities in Bangladesh's low-lying coastal regions have
always been vulnerable to catastrophic flooding events, this seems to be
happening with growing frequency. Is climate change increasing the
occurrence of flooding in Bangladesh?

We often use the Poisson distribution to model (rare) climate events
such as earthquakes and hurricanes. So let \(X_t\) be the number of
major floods in Bangladesh in time period \(t\), and let \(X_t\) be
distributed: \[ X_t \sim \text{Poisson}(\lambda) \]

\begin{enumerate}
\def\labelenumi{\alph{enumi}.}
\item
  We observe the following number of floods in Bangladesh per five-year
  period for the first quarter of the 21st century: \[
  \left[    \begin{array}{cc}
      1  & 2000-2004 \\
      3  & 2005-2009 \\
      1  & 2010-2014 \\
      2  & 2015-2019 \\
      0  & 2020-2024
   \end{array} \right] 
   \] Please write down the likelihood of this series of events for some
  unknown \(\lambda\), assuming the floods in each period are
  independent and identically distributed.
\item
  Take the log of the likelihood you wrote down in part (a). Show all
  steps.
\item
  Maximize the log-likelihood from part (b) to derive an MLE estimator
  for \(\lambda\). Show all steps.
\item
  Interpret the \(\hat{\lambda}\) you found in part (c) in your own
  words. What is this quantity conceptually, and how do you get it from
  the data?
\item
  Show that you found the MLE by plotting the log likelihood on the
  y-axis against a series of candidate values for \(\lambda\) ranging
  from 0 to 4 on the x-axis.
\end{enumerate}

\subsection{3. Bayesian Analysis}\label{bayesian-analysis}

You monitor the presence of a blue--green algae species across
freshwater sites. In a sample of \(n=274\) sites, you observe algae
present at \(y=44\) sites. Let \(\theta \in (0,1)\) denote the true
probability that a randomly selected site has detectable
algae.\footnote{This question is adapted from
  \url{https://avehtari.github.io/BDA_course_Aalto/assignments/assignment2.html}.}

\begin{enumerate}
\def\labelenumi{\alph{enumi}.}
\item
  Assume: \[
  y \sim \mathrm{Binomial}(n,\theta), \qquad \theta \sim \mathrm{Beta}(\alpha,\beta).
  \] Take as a baseline prior \(\alpha=2,\ \beta=10\). Using
  Beta--Binomial conjugacy, write down the posterior
  \(p(\theta\mid y,n)\) and identify its parameters.
\item
  Give the expression (in terms of \(\alpha,\beta,y,n\)) for the
  posterior mean of \(\theta\).
\item
  Alternatively, we may consider using the priors below: \begin{align*}
  &\mathrm{Beta}(1,1)\quad\text{(uniform)} \\
  &\mathrm{Beta}(0.5,0.5)\quad\text{(Jeffreys-type weak prior)} \\ 
  &\mathrm{Beta}(100,2)\quad\text{(strongly informative, favoring large \(\theta\))}
  \end{align*} Please plot, on a common \(\theta \in [0,1]\) axis, the
  posterior densities for the four priors (the baseline prior in part
  (a) and the alternative priors above).
\item
  In a few sentences, interpret how prior shape and strength influence
  the posterior relative to the data. Which prior(s) seem the most
  defensible in this context? If you were interested in monitoring algae
  presence, what would be your takeaway from this analysis?
\end{enumerate}

\end{document}
